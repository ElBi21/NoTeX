\chapter{NoTeX}

\noindent This is an explanatory chapter with most of the things that you will find in this template. Let's start, shall we?

\section{Boxes}

\noindent There are multiple boxes that you can use. The ones here shown are the ones that are actually working as of the current release. In the code there are also other ones, but they are not really working. They will be fixed in the future.

\begin{question}
    But how many boxes do we have? They can be divided into \textbf{two main macro groups}: \textbf{theory boxes} and \textbf{practice boxes}. This, for instance, is a box made for underline a special statement, or an important question.
\end{question}

\subsection{Theory Boxes}

\begin{definition}{NoTeX}
    Is a template
\end{definition}

\begin{corollary}{NoTeX}
    NoTeX is licensed
\end{corollary}

\begin{theorem}{NoTeX}
    NoTeX is a nice template
\end{theorem}

\begin{lemma}{NoTeX}
    NoTeX is written in \LaTeX
    \\
\end{lemma}

\begin{proof}{lemma}
    Don't you see the code? Geez
\end{proof}

\subsection{Practice Boxes}

\begin{example}
    An example of a NoTeX box
\end{example}

\begin{exercise}
    Try to edit some colors!
\end{exercise}

\begin{curiosity}{NoTeX}
    NoTeX was done in 2023! It's pretty recent
\end{curiosity}

\begin{remark}{NoTeX}
    We can always do better. NoTeX can be better too!
\end{remark}
