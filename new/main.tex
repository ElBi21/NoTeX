\documentclass[]{notex}

\usepackage{blindtext}
\usepackage{makecell}
\usepackage{tikz}

\newtcolorbox{doccommand}[1]{
    colback = maindoccol
}
\newcommand{\becomes}[1]{\hspace{#1} $\--$becomes$\rightarrow$ \hspace{#1}}

\title{NoTeX}
\author{ElBi21 - Leonardo Biason}
\date{Somewhere in 2024}

\begin{document}

\maketitle

\chapter{Introduction}

\noindent This is an attempt to transform the old project of NoTeX into an actual class

\chapter{The class}

\section{Commands of the class}

This class provides various commands, which allow to customize to your liking the template. Here is a list of all the supported (and provided) commands:

\begin{tcolorbox}
    \verb|\maincol{format}{code}|
    \tcblower
    \begin{tabular}{r l}
        \textbf{Examples}: & \verb|\maincol{HTML}{303342}| \\ 
                           & \verb|\maincol{RGB}{30, 25, 76}|
    \end{tabular}
\end{tcolorbox}

\noindent Sets the main color for the document.
\begin{itemize}
    \item [\texttt{format}] specifies the format of the color. Supports all the formats of the \texttt{xcolor} package;
    \item [\texttt{code}] is the code of the color. Must match the format given by the previous parameter
\end{itemize}

\subsection{First part}

placeholder

\chapter{The \texttt{notexmacro} package}

\noindent In this chapter the \texttt{notexmacro} package will be explained. Such package contains a collection of macros which can turn to be useful while using the class. The package is automatically required by the class, so you don't have to install anything else.
\\
\indent The macros are divided into two groups: the \textbf{math} macros and the \textbf{text} macros. The \textbf{math} macros are used within a math environment, while the text macros are not. There are some \textbf{general purpose} macros which can be used both inside and outside a \textbf{math} environment.

\section{The \texttt{math} macros}

\begin{tcolorbox}
    \verb|\eq|
\end{tcolorbox}

\noindent Adds space around a $=$. An example follows:
\begin{center}
    \verb|a \eq b| \becomes{20pt} $a \eq b$
\end{center}

\begin{tcolorbox}
    \verb|\thus|
\end{tcolorbox}

\noindent Adds substantial space around a $\Longrightarrow$, and it can be used to define a logical implication (\textit{we have $A$, thus we can get $B$}). An example follows:
\begin{center}
    \verb|a \thus b| \becomes{20pt} $a \thus b$
\end{center}

\begin{tcolorbox}
    \verb|\nextline| and \verb|\prevline|
\end{tcolorbox}

\noindent Adds a $\Longrightarrow$ which can be used at the end (with \verb|\nextline|) or at the beginning (with \verb|\prevline|) of an equation. An example follows:
\begin{center}
    \verb|ax + b \nextline| \becomes{20pt} $ax + b \nextline$ \\
    \verb|\prevline ax + c| \becomes{20pt} $\prevline ax + c$
\end{center}

\begin{tcolorbox}
    \verb|\nextline| and \verb|\prevline|
\end{tcolorbox}

\noindent Adds a $\Longrightarrow$ which can be used at the end (with \verb|\nextline|) or at the beginning (with \verb|\prevline|) of an equation. An example follows:
\begin{center}
    \verb|ax + b \nextline| \becomes{20pt} $ax + b \nextline$ \\
    \verb|\prevline ax + c| \becomes{20pt} $\prevline ax + c$
\end{center}

\end{document}